% LaTeX template for Artifact Evaluation V20201122
%
% Prepared by 
% * Grigori Fursin (cTuning foundation, France) 2014-2020
% * Bruce Childers (University of Pittsburgh, USA) 2014
%
% See examples of this Artifact Appendix in
%  * SC'17 paper: https://dl.acm.org/citation.cfm?id=3126948
%  * CGO'17 paper: https://www.cl.cam.ac.uk/~sa614/papers/Software-Prefetching-CGO2017.pdf
%  * ACM ReQuEST-ASPLOS'18 paper: https://dl.acm.org/citation.cfm?doid=3229762.3229763
%
% (C)opyright 2014-2020
%
% CC BY 4.0 license
%

\documentclass{sigplanconf}

\usepackage{hyperref}

% for bash commands
\usepackage{listings}

\begin{document}


\special{papersize=8.5in,11in}

%%%%%%%%%%%%%%%%%%%%%%%%%%%%%%%%%%%%%%%%%%%%%%%%%%%%
% When adding this appendix to your paper, 
% please remove above part
%%%%%%%%%%%%%%%%%%%%%%%%%%%%%%%%%%%%%%%%%%%%%%%%%%%%



\appendix
\section{Artifact Appendix}

%%%%%%%%%%%%%%%%%%%%%%%%%%%%%%%%%%%%%%%%%%%%%%%%%%%%%%%%%%%%%%%%%%%%%
\subsection{Abstract}

% {\em Obligatory}

Our artifacts consist of an extension to Dyninst (a famous binary transformation framework). 
We use test programs to verify correctness and measure the overhead caused by our approach - incremental CFG patching. The test
program instruments every basic blocks with empty instrumentation, which will trigger relocating all functions.


\subsection{Artifact check-list (meta-information)}

% {\em Obligatory. Use just a few informal keywords in all fields applicable to your artifacts
% and remove the rest. This information is needed to find appropriate reviewers and gradually 
% unify artifact meta information in Digital Libraries.}

A SPEC2017 benchmark suite is needed for this Artifact Evaluation.

{\small
\begin{itemize}
  % \item {\bf Algorithm: }
  % \item {\bf Program: }
  \item {\bf Compilation: GCC and binary rewriting tools.}
  \item {\bf Transformations: Binary rewriting using Dyninst.}
  \item {\bf Binary: Linux ELF.}
  % \item {\bf Model: }
  % \item {\bf Data set: }
  \item {\bf Run-time environment: Root access to Ubuntu Linux. We recommend Ubuntu Bionic 18.04.}
  % \item {\bf Hardware: }
  % \item {\bf Run-time state: }
  % \item {\bf Execution: }
  \item {\bf Metrics: We use execution time as one of the metrics.}
  \item {\bf Output: For SPEC2017 benchmark, the results are printed into a text file. For web-browser-based benchmarks, results are shown on the browser.}
  \item {\bf Experiments: Using Bash scripts and Linux commands.}
  \item {\bf How much disk space required (approximately)?: 5GB.}
  \item {\bf How much time is needed to prepare workflow (approximately)?: One hour.}
  \item {\bf How much time is needed to complete experiments (approximately)?: A day and a half.}
  % \item {\bf Publicly available?: Yes.}
  % \item {\bf Code licenses (if publicly available)?: }
  % \item {\bf Data licenses (if publicly available)?: }
  % \item {\bf Workflow framework used?: }
  % \item {\bf Archived (provide DOI)?: }
\end{itemize}

%%%%%%%%%%%%%%%%%%%%%%%%%%%%%%%%%%%%%%%%%%%%%%%%%%%%%%%%%%%%%%%%%%%%%
\subsection{Description}

\subsubsection{How to access}

The artifacts are available on GitHub, \url{https://github.com/mxz297/dyninst/tree/layout_opt}.

% {\em Obligatory}

\subsubsection{Hardware dependencies}

For x86\_64 binary rewriting, we recommend a 64GB memory.

\subsubsection{Software dependencies}

(1) For x86\_64 arch, we recommend a GCC version 7.3.0 or 7.5.0;

(2) For ppc64le arch, we recommend gcc 6.4.0; 

(3) For aarch64 arch, we recommend gcc 6.4.0.


% \subsubsection{Data sets}

% \subsubsection{Models}

%%%%%%%%%%%%%%%%%%%%%%%%%%%%%%%%%%%%%%%%%%%%%%%%%%%%%%%%%%%%%%%%%%%%%
\subsection{Installation}

% {\em Obligatory}

You can get the artifacts from GitHub and its dependencies using the following
commands.


% \noindent See the following command :
% \begin{lstlisting}[language=bash]
%   $ wget http://tex.stackexchange.com
% \end{lstlisting}

git clone https://github.com/StanPlatinum/ShadowGuard.git

cd ShadowGuard

./bazel.sh deps


%%%%%%%%%%%%%%%%%%%%%%%%%%%%%%%%%%%%%%%%%%%%%%%%%%%%%%%%%%%%%%%%%%%%%
\subsection{Experiment workflow}

The overall workflow consists of the following steps:

1. Install the artifact and dependencies;

2. Set environment variables;

3. Build the benchmarks;

4. Rewrite the benchmarks;

5. Run the benchmarks.

We provide scripts for each of the steps above.

%%%%%%%%%%%%%%%%%%%%%%%%%%%%%%%%%%%%%%%%%%%%%%%%%%%%%%%%%%%%%%%%%%%%%
\subsection{Evaluation and expected result}

% {\em Obligatory}

We will go through the entire experiment workflow by describing all the commands in each step.

We provide different rewriting modes, \textit{funcptr} mode and \textit{jumptable} mode.






\subsubsection{Micro-benchmark}

SPEC2017.

\subsubsection{Macro-benchmarks}


Anyone can build the tool using the following commands.

git clone https://github.com/StanPlatinum/BlockTrampoline.git

cd BlockTrampoline

%modify the makefile and prep.sh

./prep.sh

make

% above steps could be in one single bash script


As described in our paper, several and macro-benchmarks are tested. Here we demonstrate how Firefox and Docker binary can be rewrited by our tool.

\vspace{2pt}\noindent\textbf{Firefox}. Still, we provide two modes to rewrite the libxul.so.

Firefox (version 80.0) is usually shipped with the latest Ubuntu 18.04 dist. 
To install it manually, one can visit \url{https://support.mozilla.org/en-US/kb/install-firefox-linux} and choose the version 80.0 for this evaluation.


To instrument Firefox’s libxul.so in \textit{funcptr} mode, 

make firefox-funcptr


To instrument Firefox’s libxul.so in \textit{funcptr} mode, 

make firefox-jumptable

Please noted that binaries `libxul.so.funcptr' and `libxul.so.jumptable' can be generated at the current directory. 
Then, replace the original libxul.so with them respectively when evaluating different modes.

We provide two web-browser-base benchmarks.

\vspace{1pt}\noindent$\bullet$\textit{~Web Latency Benchmark}.

Benchmark Installation.

wget http://google.github.io/latency-benchmark/latency-benchmark-linux.zip
unzip latency-benchmark-linux.zip

Run Web Latency Benchmark.

./latency-benchmark

\vspace{1pt}\noindent$\bullet$\textit{~Jetstream2 Benchmark}.

Benchmark Installation.

Type https://browserbench.org/JetStream/ in Firefox search box.

Run Jetstream2.

Click the `Start Test' button.

\vspace{2pt}\noindent\textbf{Docker.}

Docker Installation guide can be found at \url{https://docs.docker.com/engine/install/ubuntu/}.

make docker

A new docker binary `docker.inst.bak' will be generated at the current directory. Replace the original docker binary (at /usr/bin/docker) with it.


%%%%%%%%%%%%%%%%%%%%%%%%%%%%%%%%%%%%%%%%%%%%%%%%%%%%%%%%%%%%%%%%%%%%%
\subsection{Experiment customization}

%%%%%%%%%%%%%%%%%%%%%%%%%%%%%%%%%%%%%%%%%%%%%%%%%%%%%%%%%%%%%%%%%%%%%
\subsection{Notes}

The `Image loading jank' cannot be measured in the stable versions of Firefox (79.0 and 80.0) by Web Latency Benchmark.

Exercise cautions with replacing original binaries (e.g., libxul.so and docker). Always prepare a backup for the evaluation.

% %%%%%%%%%%%%%%%%%%%%%%%%%%%%%%%%%%%%%%%%%%%%%%%%%%%%%%%%%%%%%%%%%%%%%
% \subsection{Methodology}

% Submission, reviewing and badging methodology:

% \begin{itemize}
%   \item \url{https://www.acm.org/publications/policies/artifact-review-badging}
%   \item \url{http://cTuning.org/ae/submission-20201122.html}
%   \item \url{http://cTuning.org/ae/reviewing-20201122.html}
% \end{itemize}

%%%%%%%%%%%%%%%%%%%%%%%%%%%%%%%%%%%%%%%%%%%%%%%%%%%%
% When adding this appendix to your paper, 
% please remove below part
%%%%%%%%%%%%%%%%%%%%%%%%%%%%%%%%%%%%%%%%%%%%%%%%%%%%

\end{document}
